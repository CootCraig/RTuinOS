\chapter{How does \rtos{} work?}

In a traditional sketch, the function \ident{loop} defines all actions
which have to be executed. The execution of the code is strictly
sequential. \ident{loop} may of course call any sub-routines, which may
call others, but this does not change the strictly sequential character of
the execution of the statements. Therefore it is difficult to execute
specific actions at specific points in time. For example toggle the state
of a LED every seconds. This might still be easy to do if your sketch does
nothing else -- see Arduino's standard example \ident{blink} --, but if
you're in the middle of a sketch, which e.g. transfers some data via the
USB port, it becomes ugly: You will have to merge some specific, the LED
serving statements, into your USB handling code. And the accuracy of the
yielded timing will not be perfect.

Imagine, you could simply write two sketches. The USB communication stays
as it is but a second sketch, e.g. the sample sketch \ident{blink}, is
defined at the same time and will be executed, too. Now the USB code is no
longer spoiled with double-checking the state of the LED but nonetheless
the LED will blink as desired.

This actually is what \rtos{} offers. It's however no a complete sketch
but just a function -- which can of course be stored in a separate C
source file --, which is executed in parallel. Write two such functions,
make them so called tasks and you get what you want.

How would this work? The Arduino board continues to have a single main
CPU, which is available for code execution. The trick is to use it
alternatingly to proceed with the one task and then with the other one. If
this switching between the tasks happens fast enough, than it's just the
same as if both would run at the same time -- only with limited execution
speed.

Does alternating between the tasks happen regularly? It depends. Different
patterns of alternating between tasks are possible. The most simple
pattern is to share the CPU in fixed portions between the tasks. The ratio
can be chosen. If we assume in our example that serving the USB port is
more challenging than flashing an LED, it would be reasonable to share the
CPU by 95:5 rather than by 50:50. This pattern is called round robin and
fits well if the tasks are completely independent of each other (as in our
example) and if all of them continuously require the CPU -- which is not
the case in our example!

Toggling the LED state can be done by permanently observing a watch and
switching the LED output when it reaches the next mark. A traditional
Arduino implementation of this strategy requires the CPU indeed
permanently and this is exactly how the sample sketch \ident{blink} is
implemented. In an RTOS you can do it better. Tell the system when you
want to toggle the LED state the next time and do nothing until. Your task
is inactive, does not require the CPU any more and is nonetheless executed
again exactly at the desired point in time. Two advantages arise. Your
task does barely consume CPU power and the regularity of the execution is
very good. This pattern is the appropriate solution for our example. The
LED is blinking very regularly, and nearly all of the CPU performance can
still be spend on the USB task, which therefore behaves the same way as if
there was no blinking LED at all.

The next pattern of sharing the CPU between different tasks is direct
coupling of tasks. By means of so called events a task can indicate a
specific situation to any other task, which will then react on this
situation. For example, the LED should not constantly blink. It is now
used to indicate the state of the USB communication by flashing a number
of times if a significant state change occurs. The number of flashes will
notify what happened. The LED task will now use an event to become active
and subsequently it'll use the execute-at-time pattern a number of times
to realize the sequence of flashes. The LED stays dark as long as the
event is not posted to the task. After the blink sequence the task will
again start to wait for the next occurrence of the event. The active time of
this task is still close to zero: Every time it is executed it'll just use
a few statements to toggle the LED state and to inform the RTOS about the
next condition under which to become active again. The main task, which
implements the communication can proceed nearly as if there was no LED
task. There's however an extension to its implementation. In case of a
significant status change it has to indicate the number of according
flashes and to post the event. The former can be done by a write to a
global variable and the latter is a simple call of an \rtos{} API
function. The global variable is shared by both tasks, the LED task will
read it when being activated by the event.

If there are more tasks the scenario becomes more complicated and we need
a new term. If two tasks tell the RTOS the time they want to be activated
again, there is a certain chance, that it is the same time. In this
situation, if this point in time is reached, \rtos{} decides that both
tasks are due -- but only one of them can get the CPU, i.e. can be
activated. Which one is decided by priority. The priority of a task is a
static, predetermined property of a task. At compilation time, you will
decide which of the tasks has the higher priority and which is the one to
get the CPU in the mentioned situation.

Similar: The tasks do not specify the same point in time but nearly the
same. Obviously, the task becoming due earlier will get the CPU. And when
the other one becomes due a bit later it might still desire to have the
CPU. Will it continue to have it? It depends again on the priority of the
tasks. If the later due task has the higher priority, it'll take over the
CPU from the earlier task. The earlier task is still due (as it didn't
return the CPU voluntarily so far) but no longer active. The later task is
both, due (it's point in time has reached) and active (it got the CPU).

When the active, later task tells \rtos{} to not longer need the CPU, the
earlier still due task will again get the CPU and continue to execute.

If a task tells the system to no longer require the CPU (by notifying:
"Continue my execution at/when") we say it is suspended. If it becomes due
again, we say it is resumed. If it is executed again, we say it is
activated.

There's always one and only one active task. What if all tasks in the
system did suspend themselves because they are waiting for a point in time
or an event? Now, there is a single task which must never suspend itself.
This task is called the idle task and \rtos{} can't be compiled without
such a task being present.\footnote{Caution, if the idle task would ever
try to suspend itself, a crash of the system would result.} The idle task
doesn't need to be defined in the code. \rtos{} uses the function
\ident{loop} as idle task. All the code in function \ident{loop} makes up
task idle. If you don't need an idle task (as all your code is time or
event controlled) just implement \ident{loop} as an empty function body.

Not implementing the idle task is however a waste of CPU time. \rtos{} spends
all time in \ident{loop} when none of the other tasks is due. So if there
are any operations in your application which can or should be done
occasionally it's good practice to put them into the idle task. There's no
drawback, if any task needs the CPU, idle is just waiting until the task
has finished. The execution speed of task idle is directly determined by
the CPU consumption of the other tasks. Idle will never slow down the
tasks, but the tasks will slow down idle. Idle is a task of priority lower
than all other priorities in the system. Idle is a task which is never
suspended but always due and sometimes active.


\section{Implementation of \rtos' Scheduler}
\label{secScheduler}

The set of rules how to share the CPU between the different tasks is
called the scheduler. Actually, \rtos{} is nothing else than the
implementation of a scheduler.\footnote{From the world of personal
computers you will associate much more with the term operating system than
just a task scheduler. In fact, in this environment, the scheduler is just
the most important part of the operating system, therefore referred to as
kernel, but surrounded by tons of utilities, mostly to support various I/O
operations. In the embedded world a real time operating system typically
doesn't offer anything else than a scheduler and so does \rtos{}.}
Understanding the details of the decision rules implemented in \rtos{} is
essential for writing applications that behave as desired. (An RTOS can
easily show effects which are neither expected nor desired.) The rules of
\rtos' scheduler will become clear in the following documentation of its
implementation.

In \rtos{}, a task is represented by a task object. All task objects are
statically allocated, there's no dynamic creation or
deletion.\footnote{Please, refer to section \ref{secOutlook} for more
detailed considerations about this.} The objects are configured in
Arduino's function \ident{setup} and stay unchanged from now (besides the
continuous update of the runtime information by the scheduler, see below).

\rtos{} manages all tasks in lists. Please refer to figure \ref{figRTuinOSScheduler} on page
\pageref{figRTuinOSScheduler}. There's one list per priority. All
tasks having a specific priority form a priority class and this class is
managed with the associated list. The number of different priorities is
determined at compile time by the application.

An additional list holds all tasks (of any priority) which are currently
suspended.

Within a task object there are a few members which express a condition
under which a suspended task is resumed. (These members are empty or in a
don't care state when a task is due or active.) More concrete, \rtos{}
knows about a limited number of distinct events\footnote{For good reasons
the events are implemented by bits in an unsigned integer word. Currently,
type \ident{uint16\_t} is used as it is a good trade off between number
of events and performance. A change to \ident{uint8\_t} or
\ident{uint32\_t} is possible but not trivial as assembler code is
affected. We thus have 16 such events.} and the mentioned condition is a
sub-set of these plus the following Boolean choice: will the task be
resumed as soon as any event of the sub-set is seen or does the task stay
suspended until all events have been seen? Moreover, three of the known
events have a specific meaning; they are all timing related and have a
parameter of kind \emph{when}.

At system initialization time all tasks are put in state suspended.
Consequently, the initial resume condition is part of the task
initialization. Typically, this condition is weak, like "resume
immediately". However, a delayed resume or a resume by an application
event is also possible. Starting some tasks with differing delays may even
be advantageous in order to avoid having too many regular tasks becoming
due all at the same time.


\section{Time based Events}
\label{secTimeBasedEvents}

The most relevant events of \rtos{} are absolute and relative time events.
An absolute time event means "resume task at given time". A relative time
event means "resume task after a given time span counted from now". You
see, the latter -- probably with time span zero -- is the most typical
initial resume condition for a task. It'll start immediately.

In the implementation a time event results from a system timer tic. The
core of the scheduler is a clock based interrupt. In \rtos{} this clock
has a default clock rate of about 2~ms, but this can be altered by the
application code. In the interrupt service routine (ISR), a variable
holding the absolute, current system time is incremented. (The rate of the
interrupt is the unit of all timing operations regardless of the actual
physical value.) The new value of the system time is compared against the
\emph{when} parameter of all suspended tasks which suspended with a
resume-at condition. If there's equality, the absolute time event is
notified to this task. For all tasks which had suspended themselves with a
resume-after condition, the \emph{when} parameter of this condition is
decremented. If it reaches null, the desired suspend time has elapsed and
the relative time event is notified to the task.

The third time based event is available only if the system is compiled
with round robin feature. Now a task may have a time slice defined. The
time slice is the maximum time the task may continuously be active. This
event is implemented directly, not as a bit, not as notification to the
task. A round robin task has a counter which is loaded with the time slice
duration at activation time. In each system timer tic it is decremented
for the one and only active task (if it only is a round robin task). When
the counter reaches null it is reloaded and the task is immediately taken
from the head of its due list and put at the end of this list. This means
the task stays due but will become inactive. Another task, the new head of
the list, is a more promising candidate for the new, active task.

The next step is to check the conditions of all suspended tasks. For each
such task it is checked if its resume condition is fulfilled, i.e. if all
events it is waiting for have been posted to it meanwhile. If so, it is
taken out of the list of suspended tasks and placed at the end of the due
list of the priority class the task belongs to.

Now, after reordering the tasks in the several lists, the ISR finishes
with looking for the new, active task. The decision is easy. It loops
over all due lists, beginning with the highest priority. The head of the
first found non-empty list is the new, active task. If all due lists are
empty, the idle task is chosen. The selected task is made active and the
ISR ends.


\section{Explicitly posted Events}
\label{secManualEvents}

Besides the system timer tic, the scheduler becomes active in two other
situations. The first one is the event explicitly posted by an application
task. \rtos{} knows a predefined number of general purpose events, which
can be posted by one task and which another task can wait for. The latter
task suspends itself and specifies the event as resume condition. Under
application determined conditions, the
former task calls the \rtos{} API function \ident{rtos\_setEvent} and the latter
task will resume. 

In this situation it depends on the priorities of the two tasks how
\ident{rtos\_setEvent} returns. If the event-setting task has the same or higher
priority \ident{rtos\_setEvent} will immediately return like an ordinary
sub-routine. The other task becomes due but not active. If the
event-receiving task has the higher priority, \ident{rtos\_setEvent} leads to
temporary inactivation of the calling task and will return only when it is
activated again.

More in general, \ident{rtos\_setEvent} is implemented as a software interrupt
(SWI) and behaves similar to the system timer ISR. It notifies the event
to all currently suspended tasks, which are waiting for it. The rest is
done exactly as the system timer ISR does. \ident{rtos\_setEvent} checks the
resume condition of all suspended tasks. Those tasks the condition of
which is fulfilled are moved to the end of their due list. Then, the new
active task is selected. This might be the same or another task. The SWI
ends with continuing the new, active task.

As a matter of fact, a call of \ident{rtos\_setEvent} will never make the
calling task undue (i.e. suspended), outermost inactive. This is the
reason, why \ident{rtos\_setEvent} may even be called by the idle task.

Side note: There is a crosswise relationship between \ident{rtos\_setEvent} and
the suspend commands. From the perspective of the task code switching
tasks appears as follows: The suspend function is invoked but it returns
out of a call of \ident{rtos\_setEvent} of another task or out of the suspend
command of a task which became due meanwhile. If a task calls
\ident{rtos\_setEvent} it doesn't need to return but could for example return
out of the initially mentioned suspend command.


\section{Application Interrupts}
\label{secInterruptEvents}

The last situation where the scheduler gets active is an application
interrupt. By compile switch, you can bind any AVR interrupt to the
\ident{rtos\_setEvent} function. The ISR of the interrupt source will call
\ident{rtos\_setEvent}. The event which is posted is no longer a general purpose
event but dedicated to this interrupt.

The actions are exactly the same as described for \ident{rtos\_setEvent} in
section \ref{secManualEvents}. Obviously, the ISR is asynchronous to the
task execution. If the posted event makes a task due which has a higher
priority than the interrupted task, the interrupted task is made inactive
(but it remains still due) and the other task will become active.

Typically, there will be such a task waiting for the event and having a
higher priority. Thus causing a task switch. There's only one use case for
this kind of scheduler invocation: The interrupt triggers a dedicated task
-- indirectly via the hidden call of \ident{rtos\_setEvent} --, which actually
serves as interrupt handler, doing all sort of things which are needed by
the interrupt. This tasks will be implemented as an infinite
\ident{while}-loop, where the \ident{while}-condition is the suspend
command that waits for the interrupt event and where the body of the loop
is the actual handler of the interrupt.


\section{Return from a Suspend Command}
\label{secReturnFromSuspend}

An important detail of the implementation of \rtos{} is the way
information is passed back from the scheduler to the application code. The
direct interaction of the application code with the scheduler is done with the
suspend commands and function \ident{rtos\_setEvent} in the \rtos{} API.

\ident{rtos\_setEvent} takes a parameter -- the set of events to be posted to
the suspended tasks -- but doesn't return anything. Here, the only
complexity to understand is, that it won't immediately return. It triggers
a scheduler act and won't return until the calling task is the due task
with highest priority. Which can be the case between immediately and never
in case of starvation.

The suspend commands take some parameters which specify the condition
under which the task will become due again, e.g. an absolute-time-event.
While the task is suspended, the different scheduler acts repeatedly double
check whether a sufficient set of events has been posted to the task (see
above). Each posted event is stored in the task object. As soon as the
posted set suffices, the task is moved from the suspended list to the end
of the due list of given priority class. From now on, since the task is no
longer suspended, no further events will be posted to this task and nor
will they be stored in the task object. Thus, the very set of events, which
made the task due is now frozen in the task object. When the task, which
is due now, becomes the active task again, the code flow of the task
returns from the suspend command it had initially invoked. At this
occasion, the stored, frozen set of events, which had made the task due is
returned as return value of the suspend command.

By simply evaluating the return code a task can react dependent on which
events it had made due. This is of particular interest if a task suspends
waiting for a combination of events. In practice this will be most often
the combination of an application event and a relative-time event, which
this way gets the meaning of a timeout. Obviously, the task needs to
behave differently whether it received the expected event or if there was
a timeout.

By the way, what has been said for the return from a suspend command also
holds true for the initial entry into a task function. Any task is
initialized in suspended state. The first time it is released the code
flow enters the task function. What's otherwise the return code of a
suspend command is now passed to the task function as function parameter.
This way the task knows by which condition it has been initially
activated.

\section{Summarizing the Scheduler Actions}

\incFigFile{RTuinOSScheduler}{Scheduler of \rtos}{0.8}

The different actions of the scheduler are depicted in figure
\ref{figRTuinOSScheduler}. The solid arrows indicate how task objects are
moved within and across the lists in different situations. The dashed
arrows represent pointers to particular task objects.

Under all circumstances, the active task -- highlighted in green -- is the
head of the top most non-empty list, i.e. the first due task in the order
of falling priority. Because of the particular importance of this task the
scheduler permanently holds a pointer to this task object. One could say,
that maintaining this pointer actually is all the scheduler has to do.

The idle task object can be considered the only member of the due list of
lowest possible priority. This task is needed as fall back if no due task
is found in any of the priority classes and the scheduler has a constant
pointer to this specific task object. If \rtos{} is idle, both pointers have
the same value; they point to the idle task object.

Arrow 1 depicts the round robin action. Round robin activities can only
apply to the currently active task. If its time slice is elapsed, it is
taken from the head of the due list and placed at the end of this list.
Naturally -- and indicated by arrow 2 -- the next object in the list
becomes the new head of the list and will therefore be the new active
task.\footnote{This is not fully correct: If the round robin action takes
place in the due list which has not the highest priority, it can
occasionally happen that a task of higher priority becomes due -- and
active -- in the same timer tic.} If all tasks in this list are configured
to have time slices (and if there were no other resume conditions), the
list is cycled and all tasks get the CPU for a predefined amount of time.

Arrow 3 shows the effect of a suspend command. If the active tasks issues
such a command, it is moved from the head of its due list to the end of
the list of suspended tasks.\footnote{It's just for simplicity of the
implementation that we place it at the end. The list of suspended tasks is
not ordered; actually it is not a list but a pool of task objects.} Again,
arrow 2 shows how the next task in the due list will become the new head
and active task. However, if there was no second task in this list, the
head of a due list of lower priority would become the new active task. In
our figure, this could then be the head of priority list 1.

Arrows 4 and 5 show how a suspended task becomes due again. In any call of
either the system timer ISR or \ident{rtos\_setEvent} all suspended tasks are
checked if their resume condition became true. If so, the task object is
moved from the list of suspended tasks to the end of their due list. The
due list a task belongs to is predetermined at compile time, when the task
priority is chosen. Actions 4 and 5 might appear in the same timer tic or
call of \ident{rtos\_setEvent} or in different ones.

Arrow 5 shows the resume of a task of the highest known priority: Here, we
have an example where an event causes the interruption of a running task.
The due list of the resumed task was empty before this action. Thus the
resume creates a due task object of new maximum priority -- higher than
that of the task which was active so far. This less prior task is
inactivated. From the perspective of the task execution its code flow is
interrupted. Note, that the inactivated task object is not moved! It
remains the head of its due list. The task is inactive but still due and
still the first candidate for reactivation within its priority class.


\section{Task Switches}

The basic principle of the scheduler is to switch between different tasks.
The chapters before explained the rules the scheduler applies to decide
when and why to switch to which task. This section explains how a task
switch is done.

The CPU doesn't know about tasks. It is a state machine, which has a set
of registers, which determine the next action in the next tic of the CPU
clock. The most relevant register in this context is the program counter
(PC). It is a pointer to one specific word in the program memory (flash
ROM). The word it points to is the next CPU command, which will be
executed. This next command will probably have an effect on one or more of
the CPU registers, it might e.g. command the CPU to add two data
registers, to add a constant to one of the data registers, to read a
memory cell into a register or to write a register into a memory cell.
While the command is being executed the PC is incremented so that it
points to the next word in the program flash. The next command is
selected. And so on. A program is build from hundreds and thousands of
these elementary commands. The sequence of commands form the program.

Since they are not endless it's possible to have several such command
sequences one after another in the program memory. These different
sequences now form the program code of the different tasks. The main thing
to do when switching between tasks is to alter the value of the program
counter from the one sequence to one of the others. There are dedicated
CPU command to do so, like \ident{jump}, which directly loads the PC with
the desired target address or \ident{rts} and \ident{reti}, which load the
PC from the stack.

Obviously, a scheduler needs to be able to later continue the left command
sequence, or task respectively. The continuation needs to seamless, to be
done as if there had not been a switch to another sequence at all. This
means in the first place, that we remind the value the PC had,
before we forced it to point to the other program sequence. But it also
means that all the CPU's data registers have exactly the same value they
used to have before the switch. This condition can only be fulfilled if we
save the contents of all registers prior to switching to another command
sequence.

The AVR CPU has 32 general purpose data registers plus a status register.
The scheduler holds a task object for each task. It would be straight
forward to have an array \ident{uint8\_t saveReg[32+1]} as member of the
task object and to place the register contents here. Yes, this would
basically work well but there is a much simpler and cheaper way to do: All
the registers are pushed onto the stack. The ease of doing starts with
saving the PC: As we saw, a task switch is always initiated by either an
interrupt or the call of an API function (a suspend function or
\ident{rtos\_setEvent}). Both machine operations start execution with pushing
the current PC onto the stack. As we intend to use the stack as storage
location of all our registers it's fine to already have the PC here, where
we want it to have. Saving the other registers on the stack is as easy:
The command set of any CPU contains a push command which directly stores
the contents of a data register onto the stack. For an AVR CPU this
command is called \ident{push}. Consequently, all interrupt service
routines and all API functions, which could initiate a task switch, start
with 32 push operations to save the general purpose data registers (an
important exception is mentioned later). This code sequence is completed
by two simple CPU commands which also push the CPU's status register onto
the stack.

The register pushing entry-commands of an ISR or a task switch causing API
function basically enable this function to change the PC to point to the
code of another task (and thus to continue with that task). Figuring out
whether this is necessary will be the next step of these functions. Quite
often this will not be the case. In which case the function terminates by
doing all in reverse order. All registers are read and taken from the
stack again. (They are "popped" from the stack.) This includes the PC as
very last register -- and the code execution continues where the interrupt
had interrupted the task code, or behind the API function respectively.
This is like any ordinary interrupt service routine.

Saving the registers on the stack is easy but only half the battle.
Different tasks can't operate on the same stack. A stack is the
implementation of the paradigm of the strictly hierarchical concept of
function calls and nested sub-function calls and all the local data of
these nested function invocations. This paradigm is not applicable to
multi-tasking. Tasks can be switched disregarding function entry and exit
points; the activated task is not a sub-function of the left other
task.\footnote{Prove: To continue the left task it is not a prerequisite
that the activates task terminates -- in order to "return" to the left
task.} Therefore, each task has its own, dedicated stack. Before switching
from one task to another we need to save the stack of the left task. At
runtime, i.e. beyond memory allocation and stack initialization
considerations, the stack actually is completely represented by the
pointer to its top -- and this "stack pointer" (the last CPU register to
mention) is what we need to save also. Here, we use an absolute storage
location. The task object has a member to hold the stack pointer of this
task. The value of this object member is meaningful always and only while
the task is inactive.

Before we proceed to the perhaps most difficult to understand detail, we
will shortly summarize what we saw so far:

All code, which can initiate a task switch is a function. Either an ISR or
a dedicated API function. All of these functions are well prepared to do a
task switch as they push the PC, all data registers and the status
register onto the stack of the active task. We say, they save the "CPU
context" onto the task's stack. And in case a task switch should become
necessary these functions know about a well-defined memory location where
to also save the stack pointer of the currently active task.

Now we reach the maybe most tricky detail, easy on the one hand but
nonetheless a bit difficult to see on the other hand. Let's put it into a
question: How could we ever get here to the point where we are about to
leave the currently active task? The answer: By the same kind of task
switch from another task into this task! Consequently, the other task will
have done the same. In this instance, there will be its dedicated stack
and this stack will have all the saved registers of that task on its top
(including the PC). So, if we load the stack pointer of that task, we just
have to do the same as if we'd return from the task switch causing
function back to the same task (see above) -- pop all registers and go
ahead with that task.\footnote{The program counter is popped as the last
register with a command \ident{reti}, "return from interrupt".}

Important to see: All tasks which are not currently active have been
deactivated the same way; all have left a stack with all registers
including the PC on its top and all their stack pointers have been saved
at the well-defined, known location. A picture of a task switch could be
as follows: Pushing the registers on the stack is like climbing up a hill,
from its top we jump to the top of any of the other hills around and then
we slide it down (i.e. restore the registers with the contents of this
other hill).

Doing the good preparation of storing the CPU context onto the stack the
complete task switch reduces to nothing else than exchanging the value of
the CPU's stack pointer. The value of the still active task is stored and
the value of the newly activated task is loaded -- that's all. Then we
leave the task switch causing function and its simple register popping
exit code loads the earlier saved context of the new task just like that
into the CPU.

The general story is complete if we mention the initialization. Before the
first task becomes active the still un-threaded initialization code of
\rtos{} prepares the stack areas of all tasks as if these tasks had been
active before and as if they had then been suspended. This is done by the
simple, assembler code free C function \ident{prepareTaskStack}. It places
33 bytes at the beginning of this memory area -- and these bytes will
become the initial values of the data registers of the CPU when the task
is activated the very first time (general purpose and status registers).
And in front of these 33 bytes it places the two or three bytes of the
wanted PC (depending on the type of AVR \uC). At runtime this is the
program memory location where to continue a task; now at initialization
time it's obviously the start address of the task or -- in C -- the
pointer to the task function. The address where the last byte has been
placed is the value of the stack pointer that has to be stored in the task
objekt.

When all stack areas are prepared the system timer interrupt is enabled
and the first task activation a little bit later can in no way be
distinguished from any later task switch.


\subsection{Interrupt versus API Functions}
% %TODO r24/r25 to push or not to push, eventMask as flag what to do, state
%chart of task activity states

Basically all task switch causing functions behave as described in the
previous section. Task switch causing functions are either interrupt
service routines or API functions. Interrupts can be the system timer tic
or an application interrupt. The API functions are either the suspend
command \ident{rtos\_waitForEvent}\footnote{This includes the derived
suspend commands \ident{rtos\_delay} and
\ident{rtos\_suspendTaskTillTime}.} or function \ident{rtos\_setEvent} to
post an event.

There's a significant difference between interrupt service routines and
API functions. The API functions can return a result to the calling code
but an ISR won't. When the ISR returns to a task (the one it had
interrupted or another one in case of a task switch) all data registers
need to be in the exactly same state as at the point of inactivation of
this task. When API functions return, some registers may be altered and
some others need to be altered. Data registers which pass parameter
information to the function could be changed in the function without
confusing the calling code after return. These "may be changed" conditions
are disregarded by \rtos; the registers will be restored even if this is
not essential. Some other data registers, which are selected by the
compiler to pass the function result back to the calling code however need
to be altered; just restoring them like the other registers would mean not
to return a meaningful function result. We only have the case of a
\ident{uint16\_t} function; the GNU compiler lets such a function return
its result in register pair r24/r25.

Consequently, if a task switch activates a task, which had been
deactivated by a suspend command, the context restore operation differs.
All registers but the register pair r24/r25 are restored and r24/r25 is
loaded with the function result. The final command \ident{reti} then pops
the PC from the stack and the task is continued (and will probably
evaluate the function result in r24/r25 as one of its first operations).

Some details still need explanation. First the stack balance. Since we
save the CPU context on the task stack we must always push and pop the
same number of bytes at each task switch. When restoring the context on
return to an originally suspended task we will not pop r24/r25 from the
stack (but load it with the function result) and therefore we must not
push these registers when saving the context. The context save operation
on entry into a suspend command will save all registers but r24/r25 -
anticipating that these registers will not be restored later.

The task switch causing API function \ident{rtos\_setEvent} is a void
function. It doesn't return a result and all registers are thus saved on
entry into \ident{rtos\_setEvent} and restored on return to the calling
task. (Which might be immediately on exit from \ident{rtos\_setEvent} or
later because of suspending another, intermediately executed task of
higher priority.)

A second, less important detail is about how the register pair r24/r25 is
loaded with the function result. If the task switching code recognizes
that the task to be activated now had been suspended before it pushes the
function result onto the stack as a last action before leaving the
function. The code sequence to leave the task switch causing function (be
it an ISR or an API function) can now be identical under all conditions: It
pops \emph{all} data registers, it pops the status register, it finally
pops the PC and it'll continue with the new active task. This simple
implementation idea requires that the save-context code pushes the
registers not in their natural order, r1\ldots r32, but the registers
r24/r25 need to be pushed as last registers.

TODO: Show flow charts for ISR, setEvent and waitForEvent

The next, important detail is about how to know whether an activated task
had originally been suspended or not. The consideration of this detail
also answers the question where the function result of the suspend
commands comes from.

TODO: Continue with this section. Place a figure of the state machine of a
task


\chapter{The API of \rtos}
\label{secAPI}

This chapter introduces the different functions of \rtos' application
programming interface (API). The intention is to explain the meaning and
use cases of the different API function, not all the details of the
function signatures. Please refer to the doxygen documentation which
directly builds on the C source code for details on function parameters,
return values, side effects, etc.


\section{Configuration of \rtos: rtos.config.template.h}

Building an \rtos{} application starts with the configuration of the
system. All elements of \rtos{}, which require application dependent
configuration have been placed in the file \ident{rtos.config.template.h}.
This file is not part of the build, it's just a template for the file which
will actually by used. Copy the template to your application source code
and rename it to \ident{rtos.config.h}. This is the name it has to have as
this is the name used in the \ident{\#include} statements in the code.
Open the renamed file in a text editor and read the comments. You'll find
a number of \ident{\#define}'s which you need to adjust according to the
demands of your application. The number of tasks (besides the idle task)
you're going to use and the number of priority classes they belong to are
the most evident settings. You may enable the round robin strategy (by
default it's turned off), you may configure your application interrupts
and you may chose the resolution of the system timer. Please find a more
in detail discussion of some of these topics below and in the comments in
the source code.

\section{Initialization of \rtos: \ident{setup}}

The \rtos{} application starts with a call of \ident{setup}. This function
is a callback from the Arduino startup code (and the \rtos{} initialization
code at the same time) into your application. If you do not provide this
function the linker will report a problem and refuse to build the
executable. You may put all the initialization code of your application
here. And you need to place the specification of the tasks of your
application here. No other code location is possible to do this.


\section{Specification of Tasks: \ident{rtos\_initializeTask}}

From within \ident{setup} you have to call \ident{rtos\_initializeTask}
once per task. This function specifies the properties of a task. It's
important to know, that the specific \rtos{} interrupts have not been
started while \ident{setup} is executed. You may thus interfere with any
data objects owned by your tasks without consideration of race conditions
and access synchronization.

The most important thing to specify is the executable code of the task,
i.e. the task function. A specific function pointer type,
\ident{rtos\_taskFunction\_t}, has been defined for this purpose. A
task function is a void function with a single parameter. This parameter
will pass the set of events to the task code, which made the task due the
very first time after system startup; in the most typical use cases this
will be just one of the two system timer events and may be ignored by the
task code.

Typically, task functions cycle around (similar to the function
\ident{loop} in a traditional Arduino sketch). Some other RTOS offer a
task termination but \rtos{} doesn't. In \rtos{} a task needs to cycle
around and it must never end. If it ever did it would "return" to the reset
vector of the Arduino board and your application would start up again --
and probably stay in a loop of repeatedly doing so.

The stack area of the task is specified by a pointer to a reserved address
space in RAM and by the length of this reserved area. \emph{Reserved} means
that your application has to allocate the necessary memory. Since the
stack area can never be changed at runtime it doesn't make sense to
consider some dynamic allocation operations. Just define an array of
\ident{uint8\_t} and pass its address and size.

Caution: Reserved also means that the specified stack area must never be
touched by your application code.

If you configured \rtos{} to support the round robin strategy, you will
specify the duration of the time slices the task gets.

Another important part of the task specification is the initial resume
condition. It is specified exactly like at runtime when using the suspend
command \ident{rtos\_waitForEvent}; please refer to section
\ref{secAPIWaitForEvent} for details. Your task will only start up if the
condition specified now becomes true.

\ident{rtos\_initializeTask} uses an index to identify which task object is
meant by the function call. The index has no particular meaning besides
being a kind of handle to the same task object when later using the
diagnostic functions \ident{rtos\_getTaskOverrunCounter} and
\ident{rtos\_getStackReserve}. The index needs to be lower than the number
of tasks but will not have an impact on the priority or order of execution
of the tasks.


\section{Initialization of Application Interrupts:
\ident{rtos\_enable\-IRQ\-User\-\textless{}nn\textgreater}}

Your application may configure \rtos{} to use its own interrupts which
trigger dedicated tasks. If you configure an interrupt a callback into
your application code is made as part of the system startup. The function
\ident{rtos\_enableIRQUser\textless nn\textgreater} is intended to
release your interrupt \textless nn\textgreater. This
typically means to configure some hardware registers of a peripheral
device and to finally set the so called interrupt mask bit.

You must never try to do this as part of the general initialization code
in \ident{setup}: At this point in time the task, which is coupled to the
interrupt is not ready to run and releasing the interrupt now could cause
unpredicted behavior of the service routine of your interrupt.

On the other hand, when \ident{rtos\_enableIRQUser\textless
nn\textgreater} is invoked, the \rtos{} system is up and running (besides
your interrupt) and all implications with respect to race conditions and
data access synchronization need to be considered.

Your application needs to enable the interrupt source but it doesn't have
to implement a service routine. This routine is part of the \rtos{}
implementation. Its standard action -- which can not be changed -- is to
post a dedicated event. Your application will surely specify a task of
high priority which cyclically waits for this event and implements the
actual interrupt service code.

The specification of the interrupt source is one detail of the \rtos{}
configuration made in \ident{rtos.con\-fig.h}.

Currently, \rtos{} implements up to two application interrupts (i.e.
\textless nn\textgreater{} is either 01 or 02), but it's simple and
straight forward to extend the implementation to more interrupts if
required.

After return from the last callback \ident{rtos\_enableIRQUser\textless
nn\textgreater} your application is completely up and running.


\section{The idle Task: \ident{loop}}

Once the system is started it cyclically calls the void function
\ident{loop} which has to be implemented by your application. If you do
not provide this function the linker will report a problem and refuse to
build the executable.

The repeated call of \ident{loop} is the idle task of \rtos{}. This means
the execution of this code is done only if no other task requests the
CPU. The execution speed of \ident{loop} is directly dependent on the
activities of your tasks. Therefore it typically contains code which has
no real time constraints.

A typical use case of the idle task is to put some diagnostic code here.
For example, \rtos{} permits to double-check the usage of the stack areas.
(If a task would ever exceed its reserved stack area an immediate crash is
probable; which is a hard to find bug in the code.) This code is quite
expensive but when located in the idle task it doesn't matter at all. The
only impact of expensive code in the idle task is that the results will be
available somewhat later or less frequently. For a diagnostic function
this is typically uncritical.

\rtos{} supports typical use cases where always at least one task is due.
An example is a couple of tasks, all continuously running, and scheduled
by the round robin strategy. They are cyclically activated but never
suspended. In this situation \ident{loop} will never be called. However,
as a prerequisite of successful code linkage it's nonetheless required to
have it.

If no idle task is required or if there's no idle time left simply
implement \ident{loop} as an empty function.


\section{Suspend a Task: \ident{rtos\_waitForEvent}}
\label{secAPIWaitForEvent}

A task in \rtos{} stays due as long as it desires. If it has finished or
if it becomes dependent on the work result of some other task or on an
external event (reported by an application interrupt) it will suspend
itself voluntarily.

In a technical system, a task is often applied to do a regular operation,
e.g. read and process the input value from an analog-digital converter
every Milli second.\footnote{In the \rtos{} default configuration the
system timer tic is about 2~ms; a one Milli second task can't be
implemented without a configuration change.} Here, "finished" would mean
having performed this action. When the value of this Milli second has been
processed, the task would suspend until the next Milli second interval
begins. Suspending always includes a condition under which the suspended
state ends -- in our example it would be the absolute timer event and its
parameter \emph{when} would be set to the next Milli second. From the
perspective of the task code flow, voluntarily suspending always means to
wait for something and doing nothing until. This explains the name of the
suspend function. With the view on the complete system, suspending means
to return the CPU and to pass it to other tasks, which currently don't
have to wait for whatever events.

In \rtos{} a task can suspend and wait
\begin{itemize}
  \item until a point in time,
  \item for a while,
  \item until a set of events has been posted by other tasks (or a timeout
    has elapsed meanwhile),
  \item until at least one event out of a set has been posted by other
    tasks (or a timeout has elapsed meanwhile).\footnote{Actually, the
    first two conditions are special cases of the last two: The set of
    events to wait for just contains a timer event but nothing else.}
\end{itemize}

The signature of the suspend command has a set of events as bit vector
(with up to 16 bits or events respectively) a Boolean operator (all events
required or any event releases the task) and a time parameter.

The time parameter doesn't care if no timer event is part of the set of
events. If the absolute timer event is in the set the time parameter has
the meaning \emph{when}. If the relative or delay timer event is in the
set the time parameter has the meaning \emph{after}. Consequently, it is
not allowed to have both timer events in the set.

\begin{lstlisting}[float, caption=Typical use case: regular task,
label=lstRegularTask, captionpos=b]
static void regularTask(uint16_t initialResumeCondition)
{
#define MY_TASK_CYCLE_TIME 1 /* Unit is system timer tic, e.g. 1 ms */
    do
    {
      /* Actual task implementation: read ADC, process value... */
      ...
    }
    while(rtos_waitForEvent( RTOS_EVT_ABSOLUTE_TIMER
                           , /* waitForAllEvents */ false
                           , /* when */ MY_TASK_CYCLE_TIME
                           )
         );
} /* End of regularTask */
\end{lstlisting}

A bit specific is the parameter \emph{when} of the absolute timer. The
most typical use case of the absolute timer event is the implementation of
a regular task; in our example above a task, which is activated every
Milli second. See listing \ref{lstRegularTask}: The implementation will
place the action into an infinite loop. The \ident{while} condition at the
end of the loop will be a call of \ident{rtos\_waitForEvent}, addressing
to the absolute timer event. In each loop the next Milli second is passed
as parameter \emph{when}. This would require an accumulating variable in
the implementation, which is updated in every loop. To avoid this, the
parameter \emph{when} is defined to be a difference, the difference to the
last recent reference to the absolute timer. This would mean in our typical
use case, that parameter \emph{when} becomes a constant. In every loop,
the parameter simply is 1~ms in the call of \ident{rtos\_waitForEvent}.


\subsection{\ident{rtos\_suspendTaskTillTime}}

To further support the typical use case of regular tasks, there's an
abbreviation of the call of \ident{rtos\_wait\-For\-Event} as sketched in
listing \ref{lstRegularTask}. Instead of calling
\ident{rtos\_waitForEvent} one can call
\ident{rtos\_sus\-pend\-Task\-Till\-Time}. The only parameter of the
function is the parameter \emph{when}.

\ident{rtos\_suspendTaskTillTime} is implemented as preprocessor macro, so
there's no difference in comparison to directly using
\ident{rtos\_waitForEvent} except for the readability of the code.


\subsection{\ident{rtos\_delay}}

Another abbreviated call of \ident{rtos\_waitForEvent} supports the
condition "wait for a while" (but not for any specific event). You may use
the preprocessor macro \ident{rtos\_delay} for this. The only macro
parameter is the timer parameter, which specifies the delay time (or the
time to stay suspended respectively).

There's no difference in comparison to directly using
\ident{rtos\_waitForEvent} except for the readability of the code.


\section{Awake suspended Tasks: \ident{rtos\_setEvent}}

Timer events are entirely managed by the system, all other events will
only occur if they are posted by the application code. This can be done
either by application interrupts or by invoking the API function
\ident{rtos\_setEvent}.

The only parameter of the function is the set of events to be posted,
implemented as a bit vector of 16 bits. Neither the timer events nor
application interrupt events must be posted; there remain (dependent on
the configuration of \rtos{}) twelve to fourteen events, which are
directly handled by the application task code.

There are predefined names for the available events, please refer to
\refRTOSH. Regardless, you may also define your own, suitable names.

A typical use case of application handled events are producer-consumer
models. One task prepares some data and signals availability to the data
consuming task by setting an event. Obviously, the consumer starts with
waiting for this event.

Be aware, an event is broad-casted only to the currently suspended tasks
and is not stored besides that. If a task suspends shortly after another
has posted an event, the suspend task will never receive this event and
may stay suspended forever.


\section{Data access: \ident{rtos\_enter/leaveCriticalSection}}
\label{secEnterLeaveCritSec}

In all relevant use cases, tasks will share some data. Some tasks will
produce data, others will read it. If your application has tasks of
different priority, this becomes a matter. Except for a few trivial
examples like reading or writing a one Byte word, all data access is not
atomic, i.e. can be interrupted by any system interrupt. The software has
to anticipate that this is an \rtos{} system timer interrupt or an
application interrupt which can easily cause a task switch. A task can be
in-activated while it is busy updating the data and another task can
continue operating on the same, half way completed data. The results are
unpredictable and surely wrong. Be aware, even an atomic looking operation
like \verb$++u;$, where \ident{u} is a of type \ident{uint8\_t}, is unsafe
and requires protection.

The pair of API functions \ident{rtos\_enterCriticalSection} and
\ident{rtos\_leaveCriticalSection} makes any portion of code which is
enclosed atomic -- and thus safe with respect to shared access from
different tasks.

\ident{rtos\_enterCriticalSection} simply inhibits all those interrupts,
which can cause a task switch and \ident{rtos\_leaveCriticalSection}
re-enables all those interrupts.

An application may implement interrupts, which can set an \rtos{} event
and cause a task switch. These interrupts are obviously relevant to
\ident{rtos\_enter/leaveCriticalSection}, they need to be inhibited also.
Consequently, if your application implements interrupts you will have to
extend the default implementation of the pair of functions. The functions
are implemented as preprocessor macros in the application owned \rtos{}
configuration file \refRTOSConfigH{} and their modification should be
straight forward.

The two functions do not save and restore the interrupt-inhibit state.
After any \ident{rtos\_leave\-Cri\-ti\-cal\-Sec\-tion}, all interrupts are
surely enabled. Therefore, the pairwise calls of the functions can't be
nested. The code in the outer pairs wouldn't be protected.

The pair of functions \ident{cli} and \ident{sei} from the AVR library
nearly has the same meaning and can also be used to make data access
operations atomic. The difference is that they inhibit all interrupts. The
responsiveness of the system could be somewhat degraded without need, e.g.
the Arduino time functions like \ident{delay} or \ident{millis} would
suffer. On the other hand, these two functions are a bit cheaper. We
suggest to use them if the protected code sequence is rather short, e.g.
just one or a few simple assignments and to use
\ident{rtos\_enter/leaveCriticalSection} otherwise.

In \rtos{} a task of higher priority will never become inactive in favor
of a lower prioritized task as long as it doesn't suspend itself
voluntarily. And if the task is not a round robin task it'll even never
become inactive in favor of a equally prioritized task (as long as it
doesn't suspend itself). Therefore,
\begin{itemize}
  \item a normal task of same or higher priority doesn't need atomic
    operations to access data it shares with other tasks of same or lower
    priority,
  \item a round robin task of higher priority don't need atomic operations
    to access data it shares with other tasks of lower priority.
\end{itemize}
But vice versa, their counterparts of same or lower priority of course need to
protect their access code to the same, shared data.

In cooperative systems task generally don't need to protect their access
to shared data as tasks will never be interrupted at unforeseeable (and
undesirable) points in time. In \rtos{} cooperative multi-tasking
applications are implemented by tasks all belonging to the same priority
class.

To summarize, 
\begin{itemize}
  \item always put your data access code into a pair of protective
    functions if the task shares this data or parts of it with at least
    one other task of higher priority,
  \item always put your data access code into a pair of protective
    functions if the task has round robin characteristics and if it shares
    this data or parts of it with at least one other task of same or higher
    priority,
  \item use \ident{rtos\_enter/leaveCriticalSection} as pair of protective
    functions if the access code is complex,
  \item use \ident{cli}/\ident{sei} as pair of protective functions if the
    access code is trivial.
\end{itemize}


\section{Diagnosis: \ident{rtos\_getTaskOverrunCounter}}

Each task has a built-in overrun counter. The meaning of this counter is
well defined only for regular tasks. These tasks want to become due at
fixed points in time. If too many tasks have too much to do it may happen
that it is not possible to make a task due at the desired point in time.
This is then a task overrun event. It is counted internally. This function
reads the current value of the counter for a given task.

Using this function, the application can write some self-diagnostic code.
However, if such events are seen, there's barely anything to do at
runtime. Evaluating the counters should be considered a kind of debug
information, a hint at development and testing time that the
implementation is still insufficient and needs changes.


\section{Diagnosis: \ident{rtos\_getStackReserve}}

A simple algorithm determines the usage of the task stacks. (In any RTOS,
each task has its own, dedicated stack.) The maximum size of the stacks is
predetermined at compile time and determining the actual stack usahe at
runtime is just a development tool. If the maximum stack size of any task
is exceeded the system will surely crash and the cause of the crash will
be hard to find. Allocating the stack sizes much too large is too
expensive with respect to RAM usage. Therefore, by applying this function,
you can keep an eye on the actual stack usage during development and
testing phase and reduce the allocation to a reasonable value.

There are two pitfalls. The algorithm searches the stack area for a
specific byte pattern the complete area has been initialized with, and
which is typically not written into the stack at runtime. However, it
could be written at runtime with a low but significant probability. As a
result, the actual stack usage can be one Byte more than computed with a
probability that must not be neglected. It's however much less probable
that two such bytes will be written subsequently into the stack at runtime
-- the probability that the computed number of bytes is too little by two
is much less. And so forth. If you add a number of five Byte to the
computed stack usage the remaining probability that this is less than the
actual stack usage is negligible.

The current stack usage increases suddenly by 36 Byte in the instance of a
system interrupt (the CPU context of the interrupted task is saved onto
the stack of this task). \ident{rtos\_getStackReserve} returns the maximum
stack usage so far (actually is returns the inverse value, the
\emph{reserve}, but this is equivalent). This is a useful value once your
testing code ran through all code paths, particularly through the deepest
nested sub-functions. You can ensure this by dedicated test routines. But
can you also be sure that a system interrupt occurred in the very instance
of being inside the deepest nested sub-routine? If not, it'll surely
happen later and another 36 Byte of stack will be consumed. It's good
practice to add another 36 Byte to the computed stack usage.

Summarizing, you should add 41 Byte to the computed stack usage before
reducing the stack size to the really needed value.


\chapter{Writing an \rtos{} Application}
\label{secHowToWriteApp}

\section{Short Recipe}

Create an empty folder in folder \ident{code\textbackslash{}applications}.
The name of the folder is the name of your application. Copy the
configuration template file \refRTOSConfigTemplateH{} into this folder and
rename it to \ident{rtos.config.h}.

Open \ident{rtos.config.h} and configure the number of tasks and the
number of different priority classes. Do not use empty priority classes,
this wastes expensive memory. Priorities should always be counted 0, 1,
\ldots{}, max. Select the word width of the system time. Often an eight Bit
value will be sufficient. Please refer to \ref{secSystemTimer}. In
general, you'll find a lot of hints and comments in the configuration file
telling you what to do in detail.

Open a new C source file in the same folder. This file implements the core
of your application. You need two standard functions, \ident{setup} and
\ident{loop}, your task functions and some static data.

Create a static array for each task. The type is \ident{uint8\_t}. This
array will become the task's stack area. As a rule of thumb a size of 100
\ldots{} 200 Byte is a suitable starting point. Later, you may apply
\ident{rtos\_getStackReserve} to get a better idea.

Create empty task functions: \verb+static void taskFct(uint16_t)+.

In \ident{setup} you will call \ident{rtos\_initializeTask} once per task.
Pass the pointer to the task function, the stack area and specify the
priority and the condition under which the task becomes initially due
(probably: immediately).

Create the empty function \ident{loop}: \verb+void loop(void)+.

\begin{lstlisting}[float, caption={Typical task, regularly activated},
label=lstTypicalTask, captionpos=b]
static void task10ms(uint16_t initialResumeCondition)
{
    do
    {
        /* Place actual task code here, e.g. the call of an external
           function. */
        myActual10MsTask();
    }
    while(rtos_suspendTaskTillTime(5 /* unit: 2ms */);
}
\end{lstlisting}


Now fill the task functions with useful functional code. Be aware,
that a task function must never be left, a system reset would be the
consequence. Therefore, you will always implement an infinite loop, e.g.
using \ident{rtos\_suspendTaskTillTime}. Find an example in listing
\ref{lstTypicalTask}. \ident{loop} may remain empty if you don't need idle
operations.

When implementing the functional code always be aware of the discussion of
protecting the access to data shared between tasks; please refer to
section \ref{secEnterLeaveCritSec}.

Compile your application using the generic makefile. If you placed all
your code in the single folder created at the beginning, all you have to
do now is to run \verb+make -s APP=myFolderName build+.

Start your application using the makefile. Your Arduino board is connected
via the USB cable. Run 
\verb+make -s APP=myFolderName COM_PORT=\\.\COM6 upload+.

The \rtos{} source file \refRTOSC{} must not be touched at all. Just open
it for reading if you what to understand how \rtos{} works.


\section{The Makefile}
\label{secMakefile}

\rtos{} as such can't be compiled, it's just a C source file which has to
be compiled with your application. (Without application code you'd end up
with unresolved externals when linking the code.) However, \rtos{} is
distributed with some autonomous test cases which are true \rtos{}
applications. All of these can be build and uploaded using the make
processor, which is part of the Arduino installation and the makefile
which is part of the \rtos{} distribution.

You can organize your applications similar to the test cases and will then
be able to use the makefile without changes for your applications also. If
your application grows and needs a more complex folder structure to keep
the source code as the test cases then you will still be able to use the
makefile, however with some simple changes.

Most often, you will do as follows: Open a Windows Command Prompt (or
Powershell window) and cd to the root directory of the \rtos{} folder.
This is where the file GNUmakefile is located. Connect your Arduino MEGA
board to the USB port, e.g. port number 6.\footnote{Other boards need some
code customization first. Please, refer to section
\ref{secArduinoBoards}.} Now type

\verb+make -s APP=tc05 COM_PORT=\\.\COM6 upload+

and test case (or application) \ident{tc05} is compiled and uploaded
to the board. The Arduino board is reset and the \rtos{} application is
started. Now you might consider to open the Arduino IDE and open the
Serial Monitor (i.e. the console window) to see what's going on. Later,
you might replace the default COM port in the makefile with your specific
port number and the command line becomes even shorter.

The makefile explains itself by calling make as follows. The command
requires that the current working directory is the root directory of
\rtos{}. Calling make with target \ident{help} will print a list of all
available targets to the console with a brief explanation:

\verb+make -s help+

An \rtos{} application can't be developed as sketch in the Arduino IDE. We
made some minor changes of the Arduino file \ident{main.c}, which would be
lost when using the IDE. You may however continue to use the IDE for
console I/O if you use the object \ident{Serial} for communication --
which is particularly useful during application development and which is
supported by the makefile process even better than by the IDE as it
enables you to conditionally have the I/O commands in the code. Their
presence may be restricted to a development compilation configuration.


\subsection{Prerequisites}

The name of the makefile is GNUmakefile. It is located in the root
directory of the \rtos{} installation. The makefile is compatible with the
GNU make of Arduino 1.0.1, which is "GNU Make 3.81".

Caution: There are dozens of derivates of the make tool around and most
of these are incompatible with respect to the syntax of the makefile. Even
GNU make 3.80 won't work with \rtos' makefile as it didn't know the
macro \verb+$(info)+ yet.

To run the make tool it might be required to add the path to the binary to
the front of -- to avoid shadowing by an incompatible derivate of make --
your Windows search path.\footnote{After installation, type make --version
to find out.} Inside your Arduino 1.0.1 installation, you'll find the make
tool as
arduino-1.0.1\textbackslash\-hard\-ware\textbackslash\-tools\textbackslash\-avr\textbackslash\-utils\textbackslash\-bin\textbackslash\-make.exe.

The makefile compiles the Arduino standard source files to the standard
library \ident{core.a}. To do so, it needs to know, where the Arduino sources
-- which are not part of this project -- are located. It expects an
environment variable \ident{ARDUINO\_HOME} to point to the Arduino
installation directory, e.g.
c:\textbackslash{}Program~Files\textbackslash\-arduino-1.0.1. You will
probably have to add such a variable to your system variables as it is not
defined by the Arduino standard installation process.

Another prerequisite of successfully running the makefile is that your
application doesn't have any two source files of same name -- although
this would be basically possible with respect to compiler and linker if
they were located in different folders.

Finally, all relevant paths to executables and source files and the file
names themselves must not contain any blanks. This includes the path to
the Arduino installation.

\subsection{Concept of Makefile}

The makefile has a very simple concept. A list of source code directories
is the starting point of all. All of these directories are searched for C
and C++ files. The found files are compiled and linked with the Arduino
library \ident{core.a}.

Post processing steps create the binary files as required for upload to
the Arduino board.

An optional, final rule permits to upload the binary code to the
Arduino board. If your application makes use of the USB communication with
\ident{Serial} all you still have to do to make your application visibly
run is an ALT-TAB to switch to the (open) Arduino IDE and a Ctrl-Shft-m to
open the console window.

On the compilation output side, for sake of simplicity of the makefile,
the folder structure of the source code is not retained. All compilation
products (*.o, among more) are collected in a single output folder. This
is the reason, why there must never be two source files of same name. Even
\ident{myModule.c} and \ident{myModule.cpp} would lead to a clash.

In the Arduino IDE the library \ident{core.a} is source code part of the
sketch. It is rebuilt from source code after a clean. The \rtos{} makefile
also contains the rules to build \ident{core.a} but it considers it a
static part of the software, which is in no way under development. It'll
be built if it's not up-to-date but it'll neither be deleted and rebuilt
in case of a rebuild (i.e. target \ident{clean}) and nor does its build
depend on the compilation configuration.

As said, the compilation is mainly controlled by a list of source code
directories. This list is implemented as value of macro
\ident{srcDirList}. The default is to have two directories: The \rtos{}
source code directory \ident{code\textbackslash\-RTOS} and a second,
variable directory. This directory is located in
\ident{code\textbackslash\-applications} but its name is provided by macro
\ident{APP}. This enables you to select an application for compilation
simply by stating \verb+APP=myRTuinOSApplication+ on the command line of
make.

If your application demands more than a simple, flat directory to manage
all its source files, you can continue to use the makefile with minor
changes: Explicitly list your source folders in \ident{srcDirList}. Now,
the macro \ident{APP} probably becomes meaningless. Consider to modify all
code locations which still reference the macro.


\subsection{Compilation Configurations}

The makefile supports different compilation configurations. On makefile
level, a configuration is nothing else than a set of C preprocessor
macros, which is passed on to the compiler. The meaning of the
configurations is completely transparent to the makefile and just depends
on the usage of the preprocessor macros in the C source code.

Two configurations are predefined (and used by the \rtos{} source code)
and any additional number of configurations can be created by a simple
extension of the makefile.

The standard configurations are called \ident{DEBUG} and
\ident{PRODUCTION}. \ident{DEBUG} defines the C preprocessor macro
\ident{DEBUG} and configuration \ident{PRODUCTION} defines the C
preprocessor macros \ident{PRODUCTION} and \ident{NDEBUG}.

Our recommendation is to place some appropriate self-diagnostic code in
the C/C++ source code, which is surrounded by preprocessor switches. This
includes diagnostic output using the \ident{Serial.println}, the USB
communication with the Windows machine, which might be useful during
development and testing phase. An example can be found in listing
\ref{lstUsageOfDebugSwitch}.
  
\begin{lstlisting}[float, caption={Usage of preprocessor switches
supporting different compile configurations},
label=lstUsageOfDebugSwitch, captionpos=b]
#ifdef DEBUG
    /* Some self-diagnostic code */
    Serial.print("Current stack reserve: ");
    Serial.println(rtos_getStackReserve(IDX_MY_TASK));
    if(rtos_getTaskOverrunCounter(IDX_MY_TASK, /* doReset */ true) != 0)
        doBlinkLED(3 /* times */);
#endif
\end{lstlisting}

A specific example of such code is the macro ASSERT, which expands to
nothing if \ident{DEBUG} is not defined (as in configuration
\ident{PRODUCTION}) but which double-checks some invariant code conditions
during development, when configuration \ident{DEBUG} is used. Please see
\refRTOSAssertH.


\subsection{Switching Applications}

Basically, switching work from one application to another just means to
change the value of macro \ident{APP} in the command line of make; but
please, in order to avoid running into severe, hard to find trouble, always
start your work on another application with a \ident{clean} or
\ident{rebuild} as target in the command line.

Why do we need a rebuild all? Any \rtos{} application will use its own
configuration; it'll have its own customized copy of
\ident{rtos.config.h}. The compilation of \rtos{} strongly depends on this
file. If you switch the application by changing the value of macro
\ident{APP} in the command line make can't safely find out that there's
another copy of \ident{rtos.config.h} in place. Therefore you need to
explicitly command make to re-compile everything.


\subsection{Selecting the Arduino Board}
\label{secMakefileSelectBoard}

The target platform is selected as value of macro
\ident{targetMicroController} in the makefile. However, not all Arduino
boards are currently supported by the implementation of \rtos{}. If you
select a \uC, which is not yet supported, you will run into error
directives in the source code. Please refer to section
\ref{secArduinoBoards} for more.

The makefile has not been tested with any Arduino boards other than the
Arduino Mega 2560. Please be aware, that additional changes on the
makefile could be necessary: Different controllers may require different
command line options of compiler, linker and flash tool. These differences
are not yet anticipated by the makefile. You need to double-check all
command lines, which are typically compiled from variables like
\ident{targetMicroController}, \ident{cFlags} and \ident{lFlags}.

You can use the Arduino IDE to find out, which command lines are
appropriate in your specific environment. In the file menu of the IDE you
can navigate to the properties dialog. Here, you can check the verbose
output for both, compilation and upload. Now select, build and upload
any of the sample sketches in the IDE. Copy the contents of the IDE's
output window and paste them into a text editor. You will find appropriate
command lines for all the tools.

There's a pit-fall: When running the flash tool avrdude the Arduino IDE
uses a protocol which unfortunately requires an additional, preparatory
reset command. This protocol works with the makefile only if you press the
reset button shortly before avrdude is run, which is at least inconvenient
if not unacceptable. The makefile uses a quite similar protocol, which
works well and doesn't require the reset. Place "-cWiring" in the command
line of avrdude.


\subsection{Selecting the USB Port}

The USB (or COM) port which is used for the connection between PC and
Arduino board has to be known by the rule, which uploads (and flashes)
the compiled and linked application. You can specify the port as part of
the command line of the makefile (type e.g. \verb+COM_PORT=\\.\COM6+).

However, in your specific environment you'll probably end up with the
always same port designation, so it might be handy to choose this port
designation as makefile's default. Look for the initial assignment of
macro \ident{COM\_PORT} in the makefile.


\subsection{Weaknesses of the Makefile}

\subsubsection{Unsafe Recognition of Dependencies}

The makefile includes the compiler generated *.d files. The *.d files
contain makefile rules, which describe the dependencies of object files on
source files. They are created as "side-effect" of the compilation of a
source file. This means, they are not present at the beginning and after a
clean and they are invalid after configuration changes which have an
impact on the actual tree of nested include statements. Particularly in
the latter situation the recognition of dependencies is not reliable and
the compilation result could be bad.

Consequently, in all cases of non-trivial changes of preprocessor macros
and include statements you should always call the rules to rebuild the
application. This is also mandatory if you switch from compiling one
\rtos{} application (or test case) to compiling another one.


\subsubsection{Building the required Output Directories}

If you compile an \rtos{} application the very first time, the directories
intended to hold the compilation products will probably not exist on your
machine yet. The makefile has a rule to create these directories but we
didn't manage to integrate these rules into the normal rule hierarchy of
compiling and linking.

Even worse, the error message you get in case of missing folders won't
lead you to the cause of the problem. Somewhat indirectly, the compiler
complains about some dependency files which can't be created. This is
caused by the file-write error because of the missing directories.

To overcome the problem you need to run the makefile with target
\ident{makeDir}. This has to be repeated for each compilation
configuration. The required directories are created and the problems
should disappear. The directories are not deleted by a \ident{clean} or
\ident{rebuild}.


\section{Usage of Arduino Libraries}

The biggest problem using \rtos{} is the lack of adequate multi-tasking
libraries. It's not generally forbidden to still use the Arduino libraries
but this introduces some risks. 

For some reasons an existing library function can not be used just like
that:

Particularly when addressing to hardware devices, strict timing conditions
can exist. In an RTOS, when a task is suspend for an unpredicted while,
this can make an operation timeout and fail. However, even in a single
threaded system, the code execution speed is not fully predictable as it
depends on compiler settings and interruptions by all the system
interrupts. Particularly, when serving the hardware device is done in a
task of high priority, the discontinuity of the code execution should not
be that much worse, that a timeout becomes a severe risk. Respectively,
time critical code should not be placed in the idle task.

Existing library functions could use static data, i.e. data which is not
local to the invocation of the function. This can be data defined using
the keyword \verb+static+ but also hardware registers, which do exist only
in a single, global instance. In this case arbitrary invocation from
different tasks will surely produce unpredictable results including the
chance of a crash. Since most Arduino library functions deal with hardware
entities, it's highly probable that they belong to this group.

Using the Arduino libraries is still possible if your tasks cooperatively
share the global entities. If there's e.g. only one task which serves the
PWM outputs and which lets other tasks access them only indirectly -- via
application owned, safely implemented inter-task communication.

A specific example is the global object \ident{Serial}, which performs
high level, stream based communication via USB. There's obviously no
chance to access this channel in an uncontrolled, arbitrary fashion by
several tasks. However, if only a single tasks does the console output at
a time it works fine, even if this task is the most heavily interrupted
idle task. By experience, \ident{Serial} is quite tolerant against timing
discontinuities of the invoking code and can be used for debugging purpose
during application development.

Even more, \ident{Serial} is implemented as blocking function; the call of
e.g. \ident{println} returns only after all characters have been processed.
Therefore the object can be successfully passed on from one task to another
after return from a \ident{println}. Any task gets its turn to write a
message. This can be done by a shared flag indicating the availability
(mutual exclusion by a semaphore) or by non-preemptive, cooperative
multitasking.

When \rtos{}' function \ident{setup} is executed, no multi-tasking is
active yet. Here you can use the Arduino libraries without more. In
particular, you can initialize all required ports and devices and the
communication with \ident{Serial}.

It's similar for library \ident{LiquidCrystal}. The implementation is
basically thread-safe; the library can be used. Of course, the tasks must
cooperatively implement a mutual exclusion when accessing the display. The
source code contains some hardware-required delays, which are of
course not implemented RTOS conform and which could be replaced by
\ident{rtos\_delay} as an improvement of the library for \rtos. 

The delay is implemented using Arduino's library function
\ident{delayMicroseconds}, which is based on a CPU consuming waiting loop.
This does not necessarily mean that task switches are inhibited or
delayed; just avoid to use the library in a task of high priority.
Nonetheless, the CPU time is lost; these function calls raise the CPU load
without need.\footnote{Arduino's delay function \ident{delay}
is different. It compares the desired return time with the
global system time, which is updated by an interrupt. If a task using
\ident{delay} is interrupted by other tasks and if the time it gets
re-activated again is ahead the desired return time it'll not continue to
loop and consume the CPU like \ident{delayMicroseconds} would.}

Most of the significant, CPU consuming delays occur in the initialization
of the display and this can be done in \ident{setup} before multitasking
actually takes place. The only other undesired delays are found in the
control commands \ident{clear} and \ident{home} -- if these are avoided or
if their waste of CPU time causes no pain there's no concern to use
\ident{LiquidCrystal} from \rtos{} tasks.\footnote{At point of release
this still needs to be proven by testing. The data is written into the
display chip by toggling one of its inputs up and down as a clock signal.
The implementation ensures a minimum time for the clock impulse. Due to
task switching it can however be arbitrarily prolonged. Whether the
hardware of the display accepts this is still unclear. If it gets
confused, the few clock commands would need to be put into a critical
section.}

Despite of all, using a library developed for single tasking in a
multi-tasking environment remains a risk, which must not be taken in a
production system. In a production system, any library function needs to
be reviewed before using it. Fortunately, all Arduino code is available as
source code, so that a code review can be done.

%DEBUG compilation and particularly ASSERT requires initialization of Serial
%Note on ASSERT: Serial, wait, reset

\subsection{Changes of Arduino's \ident{main} Function}

The file main.cpp of the Arduino standard sketch has been replaced by
main.c of \rtos. The implementation of the contained function \ident{main}
is nearly the same. The standard sketch finally enters a short, infinite
loop, which repeatedly invokes the application's "task-function"
\ident{loop}. The same loop continues to exist in \rtos{} (although it has
been moved to the end of \ident{rtos\_initRTOS}) with one major change: In
the standard sketch the loop contains the following code, which is
executed alternatingly with function \ident{loop}:
\begin{verbatim}
		if (serialEventRun) serialEventRun();
\end{verbatim}
We are not sure about the meaning of this code and which library
functionality depends on it. Moreover, we can't per se be sure that it
cooperates well with \rtos{}' multitasking and decided not to put it into
the \rtos{} code. Whoever knows to need it can try to place the statement
into function \ident{loop} of his \rtos{} application. Now, the behavior
is just the same as in the standard sketch -- besides all the obvious
differences because of task switches.


\section{Configuring the System Timer}
\label{secSystemTimer}

A central element of \rtos{} is its system time. This time is e.g. the
parameter of a suspend command if a task wants to wait until a specific
point in time (as opposed to wait for a specific while). Many of the
operations in \rtos{} and its application code deal with the system time.
Therefore, we decided to make the implementation type of the system time
subject to the configuration of the application; you have to customize the
type in your application's copy of \ident{rtos.config.h}. In many
situations a short one Byte integer will be sufficient, but not in
general. The intention of this section is to explain all implications of
the type choice to enable you to choose the optimal, shortest possible
type for your application.

The system time is a cyclic integer value. The unit is the period time of
the main interrupt, which is associated with the system timer. Each
interrupt will clock the time by one unit. If the use case of \rtos{} is a
traditional scheduling of regular tasks of different priorities, it's good
practice to choose the period time of the fastest regular task as unit of
the system time. But in general the unit of the system time doesn't matter
with respect to the function of \rtos{} and the time even doesn't need to
be regular.

You may define the time to be any unsigned integer considering the
following trade off:

The shorter the type the less the system overhead -- many operations in
the kernel are time based.

The shorter the type the earlier does the system time cycle around. This
has implications on the ratio of period times of slowest and fastest
regular task and the maximum suspend or delay time of any other task.

The longer the type the larger is the maximum ratio of period times of
slowest and fastest task. This maximum ratio is half the maximum number,
which can be represented by the chosen integer type. If you implement
regular tasks of e.g. 10~ms, 100~ms and 1000~ms period time, this could be
handled with a \ident{uint8\_t} (ratio 1:100). If you want to have an
additional 1~ms task, \ident{uint8\_t} will no longer suffice (ratio
1:1000), you need at least \ident{uint16\_t}. (\ident{uint32\_t} is
probably never useful.)

The longer the type the higher becomes the resolution of timeout timers
when waiting for events. The resolution is the clock frequency of the
system time. With a system time of type \ident{uint8\_t}, which cycles
around very soon, one would probably choose a tic frequency identical to
the frequency of the fastest task (or at least only higher by a small
factor). Then, this task can only specify a timeout which ends at the next
regular due time of the task. (Which might still be alright with respect
to error handling.)

If the clock frequency of the system time is higher than the frequency of
the fastest regular task, then the statement made before needs refinement:
The range of the data type of the system time limits the ratio of the period
time of the slowest task and the resolution of timeout specifications.

The shorter the type the higher the probability of not recognizing task
overruns when implementing regular tasks: Due to the cyclic character of
the time definition a time in the past is seen as a time in the future if
it is over more than half the maximum integer number. This leads to the
wrong decision whether we have a task overrun or not. See the example:

Data type be \ident{uint8\_t} . A task is implemented as regular task of
100 units period time. Thus, at the end of the functional code it suspends
itself with time increment 100 units. Let's say it had been resumed at
time 123. In normal operation -- no task overrun -- it will end e.g. 87
tics later, i.e. at 210. The demanded resume time is 123+100 = 223, which
is seen as +13 in the future. If the task execution was too long and ended
e.g. after 110 tics, the system time was 123+110 = 233. The demanded
resume time 223 is seen in as 10 tics in the past and a task overrun is
recognized.\footnote{When a task overrun is recognized this way the task
becomes due immediately. It is not omitted but activated significantly too
late.} A problem appears at excessive task overruns. If the execution had
e.g. taken 230 tics the current time was 123 + 230 = 353 -- or 97 due to
its cyclic character. The demanded resume time 223 is 126 tics ahead,
which is considered a future time -- no task overrun is
recognized.\footnote{One call of the task has been silently omitted and
the next one is timely again.} The problem appears if the overrun lasts
more than half the cycle time of the system timer. With \ident{uint16\_t}
this problem becomes negligible.

The idea is to choose the data type of the system time as short as
possible. Choosing the type is done with macro
\ident{RTOS\_DEFINE\_TYPE\_OF\_SYSTEM\_TIME}, please refer to
\refRTOSConfigH.


\subsection{Configuring the System Timer Interrupt}

The interrupt service routine (ISR), which clocks the system time and
which performs all related actions like resuming tasks, which are waiting
for a timer event, is a core element of the implementation of \rtos{}. The
implementation leaves it however open, which actual hardware event, i.e.
which interrupt source, is associated with the service routine. In the
standard configuration, the interrupt source is the overrun event of the
timer 2 (\ident{TIMER2\_OVF}), but this can easily be changed by the
application.

In the AVR environment, an ISR is implemented using the macro \ident{ISR}
as function prototype. A specific interrupt source is associated with the
ISR by the macro's parameter. The name of the interrupt source is stated.
(A pre-defined, \uC{} dependent list of those exists, please refer to
\refATmegaManual, section 14.) In \rtos{} the parameter of macro
\ident{ISR}, the name of the interrupt source, is implemented as other
macro \ident{RTOS\_ISR\_SYSTEM\_TIMER\_TIC},
which is defined in the application owned configuration file
\refRTOSConfigH. By simply changing the macro definition any other
interrupt source can be chosen.

Typically, a few hardware related operations are needed to make a
peripheral device a useful interrupt source. In case of timers, the timing
conditions have to be stated (how often to see an interrupt); and
generally, most peripherals require to set a so called interrupt mask bit
in order to enable it as interrupt source.

The \rtos{} standard configuration uses timer 2 as it is in the Arduino
standard configuration. Arduino uses this timer for PWM output and has
chosen appropriate settings. The only thing \rtos{} adds to the
configuration of the timer is to set the interrupt mask bit of the
overflow event. The Arduino configuration causes an overflow event about
every 2~ms.\footnote{The precise value can be found as a macro in the
\rtos{} configuration file \refRTOSConfigH. Changing the definition of
this macro belongs to the code adaptations, which are required if the
system timer interrupt is reconfigured.} This is thus the system clock of
\rtos{}.

The hardware configuration of the interrupt source is done in the void
function \ident{rtos\_enable\-IRQ\-Ti\-mer\-Tic}. The function is
implemented as a \emph{weak} function. In the terminology of the GNU
compiler this means that the application may redefine the same function.
Rather than getting a linker error message ("doubly defined symbol") the
linker discards the \rtos{} implementation and will instead put the
application's implementation in the executable code. The standard
implementation is overridden by simply re-implementing the same function
in the application code. Caution: The signature of the overriding function
needs to be identical, the type attribute \ident{weak} must however not be
used again.

The application will put all operations to configure the interrupt source
selected by macro \ident{RTOS\_\-ISR\_\-SYS\-TEM\_TIMER\_TIC} into its
implementation of the function and timer 2 will become like it used to be
in Arduino.

Another, related code modification has to be made by the application
programmer. The function pair
\ident{rtos\_en\-ter/leave\-Cri\-ti\-cal\-Sec\-tion} inhibits and
re-enables all those interrupts, which may lead to a task switch -- which
the timer interrupt evidently belongs to (see section
\ref{secEnterLeaveCritSec}). If you change the interrupt source, i.e. if
you alter the value of macro \ident{RTOS\_ISR\_SYSTEM\_TIMER\_TIC}, you
will have to modify the code of these functions accordingly. They are
implemented as macros in the application owned configuration file
\refRTOSConfigH{} and can thus be changed easily.


\section{Using Application Interrupts}

\rtos{} supports two application interrupts. The application configures
the hardware interrupt source and associates it with the already existing
interrupt service routine. The ISR itself must never be changed. It sets a
specific pre-defined event, which is related to the application
interrupt. Setting the event is done like any task could do using the API
function \ident{rtos\_setEvent}. The intended use case is that your
application has a task which cyclically suspends itself waiting for this
interrupt related event and which is hence awaken each time the interrupt
occurs. This task is than the actual handler, which implements all
required operations to do the data processing.

The application interrupt 0 is enabled in your application by turning the
preprocessor switch \ident{RTOS\_\-USE\_\-APPL\_\-INTER\-RUPT\_00} from
\ident{RTOS\_FEATURE\_OFF} to \ident{RTOS\_FEATURE\_ON}. Now, the related
ISR will be compiled and one of \rtos{}' general purpose events is renamed
into \ident{RTOS\_\-EVT\_\-ISR\_\-USER\_\-00} indicating the specific
meaning this particular event gets; it should never be set by an ordinary
task.

The existing ISR is associated with the interrupt source your application
needs by means of macro \ident{RTOS\_ISR\_USER\_00}. The value of this
macro is set to the name of the interrupt source. A table of available
interrupt sources is found in the manual of your specific controller, see
e.g. \refATmegaManual, section 14, for the \uC{} of the Arduino Mega
board.

The switch \ident{RTOS\_USE\_APPL\_INTERRUPT\_00} and the macro
\ident{RTOS\_ISR\_USER\_00} are found in the application owned
configuration file \refRTOSConfigH.

The associated interrupt source needs to be configured to fire interrupt
events. Most often, the interrupt sources are peripheral devices, which
have some hardware registers which must be configured. For example, a
regular timer interrupt would require to set the operation mode of the
timer/counter device, the counting range and the condition, which triggers
the interrupt. You will have to refer to your CPU manual to find out. All
required settings to configure the interrupt are implemented in the
callback function \ident{rtos\_enableIRQUser00}.

\ident{rtos\_enableIRQUser00} does not have a default implementation, a
linker error will occur if you do not implement it in your application
code. Caution: It is invoked by the \rtos{} initialization code at a time
when all tasks are already configured (\ident{setup} has completed) and
when the system timer of \rtos{} is already running. This means that all
multi-tasking considerations already take effect. You need to anticipate
task switches and resulting race conditions. Actually, the invocation of
\ident{rtos\_enableIRQUser00} is done early from within the idle task,
just before \ident{loop} is executed the very first time. Consider to use
the function pair \ident{rtos\_en\-ter/leave\-Cri\-ti\-cal\-Sec\-tion} to
sort out all possible problems.

\begin{lstlisting}[float, caption={Initialization of an application interrupt},
label=lstEnableAppInt, captionpos=b]
void rtos_enableIRQUser00(void)
{
    /* Inhibit all task-switch relevant interrupts. */
    rtos_enterCrtiticalSection();
    
    /* Configure the peripheral device to produce your application
       interrupt but do not enable the interrupt in its interrupt mask
       register yet. */
    ...
    
    /* Re-enable all task-switch relevant interrupts. Since you modified 
       the implementation of rtos_leaveCrtiticalSection this will also
       set the appropriate bit in the mask register of your peripheral. */
    rtos_leaveCrtiticalSection();
}
\end{lstlisting}

When using application interrupts another, related code modification has
to be made by the application programmer. The function pair
\ident{rtos\_en\-ter/leave\-Cri\-ti\-cal\-Sec\-tion} inhibits and
re-enables all those interrupts, which may lead to a task switch -- which
your interrupt evidently belongs to (see section
\ref{secEnterLeaveCritSec}). The functions need to additionally inhibit
and re-enable the interrupts you chose as source. They are implemented as
macros in the application owned configuration file \refRTOSConfigH{} and
can thus be changed easily. Please consider that
\ident{rtos\_leave\-Cri\-ti\-cal\-Sec\-tion} partly implements what
\ident{rtos\_enableIRQUser00} is expected to do, refer to listing
\ref{lstEnableAppInt} for more.

The second available application interrupt 1 is handled accordingly, you
just have to replace the index 00 by 01 in all function and macro names
referred to before.


\section{Support of different Arduino Boards}
\label{secArduinoBoards}

\rtos{} has been developed on an Arduino Mega 2560 board and this is the
only supported board so far. There are some obvious dependencies on the
\uC:
\begin{itemize}
  \item The size of the program counter is three Byte for an ATmega2560
    but only two Byte for some other derivates
  \item The availability of peripheral devices depends on the \uC{} and
    moreover,
  \item the naming of the registers may differ between \uC{}s even for the
    same peripheral.
\end{itemize}
The implementation of \rtos{} uses a preprocessor switch based on the
macro \ident{\_\_AVR\_ATmega2560\_\_} from the AVR library anywhere we have
such an obvious platform dependency. The else case is "implemented" as
error directive so that you are directly pointed to all these code locations
by simply doing a compilation with another \uC{} selected in the makefile.

All code locations, where such an error directive is placed are easy and
straight forward to modify. You will find some guidance in the code
comments close to the error directives. Nonetheless, we decided to not try
an implementation as it would be not tested.

Unfortunately, there's a remaining risk, that there are more platform
dependencies than currently anticipated in the code. This can only be
found out by doing the migration and testing. Feedback is welcome.

The makefile controlled build process depends on the Arduino board, too.
An obvious change is the specification of the \uC{} in use when compiling
the code. However, there are more changes required in the makefile. Please
refer to section \ref{secMakefileSelectBoard} for more.


\chapter{Outlook}
\label{secOutlook}

We hope that \rtos{} is deemed useful as it is and that it adds some value
to the Arduino world. Nonetheless, we are aware that it has its limitations
and that a lot of improvements are imaginable and some even feasible. Some
ideas and comments have been collected here.

The lists which hold the due tasks are implemented as arrays of fixed
length. All priority classes use a list of identical length. This has been
decided just because of the limitations of constant compile time
expressions and macros. The code would run without any change if the
rectangular array holding all lists would be replaced by a linear array of
pointers, which are initialized to point to class-individual linear arrays.
This construction would save RAM space in applications which have priority
classes of significantly differing size. Besides some saved RAM and the
less transparent initialization on application side\footnote{We don't like
to do dynamic initialization using a loop and a call of \ident{malloc}
inside in an embedded environment. This would probably consume RAM space
on the heap in the same magnitude than what can be saved by the changed
layout of the data structure.} there's absolutely no difference of
both approaches and therefore the urgency for this change is not high
enough to actually do it.

Currently, the idle task is described by an additional object in the task
array -- although it lacks most of the properties of a true task.
Actually, only the stack pointer save location and the (always zero)
vector of received events are in use. If the task object is split into two
such objects (holding the properties of all tasks and holding the
properties of true tasks only) some currently wasted bytes of RAM could be
saved.

The priority of a task could be switched at runtime if only the arrays are
large enough -- but this is anyway in the responsibility of the
application. (Condensed array implementation by pointers to
in\-di\-vi\-dual arrays became impossible in this scenario.) The
implementation is simple as it is close to existing code. The API function
would be implemented as software interrupt similar to the suspend
commands. The active task would be taken out of its class list and put at
the end of the targeted class list. The list lengths would be adapted
accordingly. Then the normal step of looking for the now most due task and
making this the active one would end the operation. \ident{rtos\_setEvent}
would probably be the best fitting starting point of the implementation.
This idea has not been implemented as we don't see a use case for it.

Currently, the round robin time (including round robin mode on/off by
setting the time to 0) is predetermined at compile time -- but without any
technical need. It would easily possible to change it by API call at
runtime. If we specify that a change won't affect the running time slice
it's very easy as the call of this function won't cause a task switch. A
software interrupt is not required, just write the reload value of the
round robin counter.

There's no strong technical reason, why a task should not end. At the
moment the return address of a task function is the reset address of the
\uC{}. By modifying \ident{prepareTaskStack} it could become any other
address, e.g. the address of a function implemented similar to the suspend
commands. It would not put the active task into the list of suspended
tasks but in a new list of terminated tasks. This list is required as task
termination makes sense only if there's also a chance to create new ones.
The list of terminated tasks would be the free-list of objects to reuse
whenever a new task is created at runtime.

Starting a new task at runtime would mean to let \ident{prepareTaskStack}
operate on a currently unused task object and to place the object in a
critical section into the list of suspended tasks. As currently, the
application is responsible for obeying the size of the lists, in
particular if there's still room in the priority class.

Since we do not want to introduce dynamic memory allocation into the
application, any started task needs to be pre-configured. It has to be
decided if under these circumstances terminating/restarting a task has a
big advantage over just suspending existing tasks permanently.

Currently, events are broad-casted only to suspended tasks. If a task
starts waiting briefly after the other task has already sent the event, it
is lost and the former task might wait forever. Two extensions are
imaginable:

\ident{rtos\_setEvent} could get a sibling, which sets a set of events and
states at the same time another set of events to wait for. The task posts
events and suspends itself with a resume condition. This function would
solve the consumer/producer problem in case of different priorities and
it's easy to implement. It's a straight forward combination of code
existing in \ident{rtos\_setEvent} and \ident{rtos\_waitForEvent}.

Secondary, events could be queued. There would be a suspend command with
the resume condition: "Let me become due again if anybody else has put an
event into the queue" - including the situation that this event had
already been put into the queue before. The implementation of this idea
would be not too difficult but it introduces completely new structural
elements like point to point communication into the implementation, which
could easily obscure the current quite concise and stringent
implementation concept. Furthermore, using
\ident{rtos\_enter/leaveCriticalSection}, an \rtos{} application can write
its own queues even though it had to use a polling strategy.

Another typical RTOS element are mutexes and semaphores. (Actually, a
mutex can be considered a special case of a semaphore, a semaphore with a
single instance.) Semaphores are useful to handle shared resources. The
difference to a critical code section is, that the access to the shared
resources can be granted to a task across task switches. Task scheduling can
go on, while a task owns the resource. A critical section generally blocks
task switches and must therefore used only on short operations. 

There's one test case of \rtos{}, which demonstrates how to implement the
idea of a mutex by a combination of flags and critical sections. Such an
implementation can indeed safely manage a resource but will always have
some weakness with respect to responsiveness and priority handling.

An implementation of true semaphores fits basically well to the existing
code. In particular if we restrict the implementation to mutexes (where
the instance counter can be replaced by a single bit in a bit vector of
\ident{n} mutexes, all handled in parallel) the implementation will
strongly resemble the current event handling and significant parts of the
existing event code could be used for mutexes without changes. Because of
this, the next release of \rtos{} will likely support at least mutexes if
not semaphores.
