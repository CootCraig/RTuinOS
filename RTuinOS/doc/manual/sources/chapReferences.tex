\chapter{References}
\label{secDocReferences}

% A list of references to each quoted or mentioned external document. In
% your document you may write "please refer to \refL2Kurz2, page 23"
% instead of writing the title. Your document will then contain "please
% refer to [1], page 23".
\def\refRTOSC{[1]}
\def\refRTOSH{[2]}
\def\refRTOSConfigTemplateH{[3]}
\def\refRTOSConfigH{[4]}
\def\refRTOSAssertH{[5]}
\def\refATmegaManual{[6]}
% TODO Add a new reference here

\begin{longtable}[c]{|c|p{5.5cm}|p{8.0cm}|}
\hline
ID & Document & Explanation \\ \hline
\endfirsthead
\hline
ID & Document & Explanation \\ \hline
\hline
\endhead
\caption[]{References (continued on next page)}
\endfoot
\caption{References} \label{tabDocReferences}
\endlastfoot
\hline
\refRTOSC & code\textbackslash\-rtos\textbackslash\-rtos.c
          & C source code file of \rtos{}
\\ \hline
\refRTOSH & code\textbackslash\-rtos\textbackslash\-rtos.h
          & Header file of \rtos{}, declaring the API
\\ \hline
\refRTOSConfigTemplateH & code\textbackslash\-rtos\textbackslash\-rtos.\-config.\-tem\-plate.h 
                        & Compile time configuration of \rtos{} (template file)
\\ \hline
\refRTOSConfigH & rtos.config.h
                & Application owned compile time configuration of \rtos{}, derived
                  from \refRTOSConfigTemplateH
\\ \hline
\refRTOSAssertH & code\textbackslash\-rtos\textbackslash\-rtos\-\_assert.h
                & Macro definition supporting code self-diagnosis
\\ \hline
\refATmegaManual & doc2549.pdf, e.g. at www.\-atmel.\-com/\-Ima\-ges/\-doc\-2549.pdf
                 & User manual of CPU ATmega2560 and others
\\ \hline
% TODO: Add a new document title here
\end{longtable}


