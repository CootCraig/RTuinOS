\chapter{Scope}
\label{secScope}

Arduino\footnote{See www.arduino.cc} is a popular \uC{} platform for
various purposes, mainly located in leisure time projects. The \rtos{}
project adds the programming paradigm of a real time operating system to
the Arduino world. Real time operating systems characterize the
professional use of \uCs{}.

Real time operating systems, or RTOS, strongly simplify the implementation
of technical applications which typically do things in a quite regular
way, e.g. checking inputs and reacting in an appropriate way every tenth
of a second. The same for the outputs, which are served in the same or
another regular time grid; this way it is pretty simple to e.g. record the
environmental temperature and to control the Arduino LED at the same time
very regularly and without the need of CPU consuming waiting loops as
in Arduinos library function \ident{delay}.

The traditional Arduino sketch has two entry points; the function
\ident{setup}, which is the place to put the initialization code required
to run your sketch and function \ident{loop}, which is periodically
called. The frequency of looping is not deterministic but depends on
the execution time of the code inside the loop.

Using \rtos{}, the two mentioned functions continue to exist and continue to
have the same meaning. However, as part of the code initialization in
\ident{setup} you may define a number of tasks having individual
properties. The most relevant property of a task is a C code
function\footnote{The GNU C compiler is quite uncomplicated in mixing C
and C++ files. Although \rtos{} is written in C it's no matter do implement
task functions in C++ if only the general rules of combining C and C++ and
the considerations about using library functions (particularly
\ident{new}) in a multithreading environment are obyed. Class member
functions are obviously no candidates for a task function.}, which becomes
the so called task function. Once entering the traditional Arduino loop,
all of these task functions are executed in parallel to the repeated
execution of function \ident{loop}. We say, \ident{loop} becomes the idle
task of the RTOS.

This document introduces the basic ideas of \rtos{} to those of you who are
not familar with the usage of an RTOS and gives anybody else an overview
what the possibilities and limits of \rtos{} are.

Some core considerations of the implementation are highlighted, but the
relevant documentation of the implementation is the code itself. It is
heavily commented using doxygen\footnote{See http://www.doxygen.org} tags.
The compiled doxygen documentation is part of this project but will
contain only the documentation of the global objects of \rtos. To
understand the implementation you will have to inspect the source code
itself, please refer to \refRTOSC.

